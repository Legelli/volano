%======================= PREAMBOLO DICHIARAZIONI INIZIALI===============%
\documentclass[10pt,oneside,a4paper]{article}

\usepackage[latin1]{inputenc} 
\usepackage[italian]{babel}
\usepackage{siunitx} %Inserisce automaticamente i dati con le unità  di misura correttamente formattate del SI (utilizzo: \SI{0.82}{m^2}, in generale \SI{misura con il punto decimale}{unità  di misura})
\sisetup{output-decimal-marker = {.}, separate-uncertainty = true, input-uncertainty-signs = \pm, detect-weight=true, detect-family=true} %per usare SI con il punto decimale
\usepackage{listings} %Per citare codice informatico formattandolo correttamente
\usepackage{amsmath}
\usepackage{graphicx}
\usepackage{geometry}
\usepackage{epigraph}
\usepackage{booktabs}	%tabelle migliorate
\usepackage{tablefootnote}	%note a piè di pagina in tabella
\usepackage{threeparttable} %tabella con note a piè di tabella
\usepackage{caption}	%descrizione per figure
\captionsetup{tableposition=top,figureposition=bottom,font=small} %setup descrizione
\usepackage{float}
\usepackage{esvect} %vettori
\usepackage{longtable} %tabelle lunghe
\usepackage[dvipsnames]{xcolor}
\definecolor{sepia}{HTML}{80002A}
\usepackage[colorlinks=true, citecolor=black, linkcolor=sepia, urlcolor=black]{hyperref}
\usepackage{mathrsfs}
\usepackage[utf8]{inputenc}


\usepackage{listings} %Per inserire codice
\lstnewenvironment{codice_c}[1][]
{\lstset{basicstyle=\small\ttfamily, columns=fullflexible,
keywordstyle=\color{red}\bfseries, commentstyle=\color{blue},
language=C, basicstyle=\small,
numbers=left, numberstyle=\tiny,
stepnumber=2, numbersep=5pt, frame=shadowbox, #1}}{}

\setcounter{section}{-1}

%========= PRIMA PAGINA ===========%
\title{\textsc{Misura della frequenza di fenomeni radioattivi e studio di un pallinometro}}
\author{\small{G. Galbato Muscio} \and \small{L. Gravina} \and \small{L. Graziotto} \and \small{M. Rescigno}}
\date{}

\begin{document}
	\begin{figure}
		\centering
		\includegraphics[scale=0.5, trim={2.8cm 8.9cm 0 9cm}, clip]{logo.png}
	\end{figure}
	\maketitle
	\begin{center} 
		\fbox{{\fontsize{12pt}{8mm}\textsc{Gruppo B2.3}}} \\
		\vspace{1cm}
		\begin{tabular}{ccc}
			Esperienza di laboratorio && Consegna della relazione \\
			\emph{\small{3 maggio 2017}} && \emph{\small{14 maggio 2017}} \\
										&& 	\emph{\small{23 maggio 2017} (seconda parte)}  \\
		\end{tabular} 
		
		\vspace{0.5cm}
		
	\end{center}
\hrule
\vspace{0.5cm}
\begin{abstract}
Utilizzando un contatore Geiger-M\"uller misuriamo la radioattività  ambientale, quella dovuta ai raggi cosmici e la radiazione emessa da un blocco di tufo. Studiamo quindi la distribuzione delle biglie in un pallinometro fisico e in uno simulato.	
\end{abstract}
\newpage
\tableofcontents %Indice
\listoftables %Indice delle tabelle
\listoffigures %Indice dei grafici
\pagebreak
\section{Convenzioni e formule}
In questa relazione verranno usate le seguenti convenzioni:
\begin{enumerate}
	\item sarà usato il punto [ $.$ ] come separatore decimale;
	\item l'approssimazione decimale della cifra $5$ sarà fatta per eccesso;
	\item al fine di snellire la relazione e migliorarne la leggibilità , riporteremo nel corpo del documento solamente le tabelle riepilogative e alcuni grafici, e dedicheremo un'appendice finale alle tabelle e ai grafici più dettagliati.
\end{enumerate}
Inoltre, si farà riferimento alle seguenti formule:
\begin{enumerate}
	\item media 
	\begin{equation}\label{eq:media}
	\bar{x} = \frac{1}{N}\sum_{i=1}^Nx_i;
	\end{equation}
	\item varianza
	\begin{equation}\label{eq:varianza}
	\sigma^2 = \frac{1}{N}\sum_{i=1}^N(x_i-\bar{x})^2;
	\end{equation}
	\item deviazione standard
	\begin{equation}\label{eq:deviazione}
	\sigma = \sqrt{\sigma^2}.
	\end{equation}	
\end{enumerate}
Le formule dei minimi quadrati utilizzate sono: 
\begin{itemize}
	\item coeficiente angolare
	\begin{equation}\label{eq:coeficienteangolare}
	\hat{m} = \frac{Cov[x,y]}{Var[x]}
	\end{equation}
	\item intercetta
	\begin{equation}\label{eq:intercetta}
	\hat{c} = \bar{y} - \hat{m} \cdot \bar{x}
	\end{equation}
	\item varianza del coefficiente angolare
	\begin{equation}\label{eq:varianzacoeff}
	Var[\hat{m}] = \frac{1}{Var[x]\cdot \sum_{i}\frac{1}{\sigma_{y_i}^2}}
	\end{equation}
	\item varianza dell'intercetta 
	\begin{equation}\label{eq:varianzainter}
	Var[\hat{c}] = \bar{x^2}\cdot Var[\hat{m}]
	\end{equation}
	\item covarianza 
	\begin{equation}\label{eq:cov}
	Cov[\hat{m}, \hat{c}] = \bar{x} \cdot Var[\hat{m}]
	\end{equation}
\end{itemize}

%===============SCOPO E DESCRIZIONE DELL'ESPERIENZA==============%
\section{Scopo e descrizione dell'esperienza}
\label{sec:descrizione}



%================APPARATO SPERIMENTALE======================%		
\section{Apparato Sperimentale}
\subsection{Strumenti}
\label{subsec:strumenti}

\begin{itemize}
	\item Volano 
	\item Sensore di posizione che misura la posizione della massa $m$ a tempi diversi;
	\item metro [divisione: \SI{0.001}{m}, incertezza: \SI{0.0003}{m}];
	\item calibro ventesimale [divisione \SI{0,05}{mm}, incertezza \SI{0,05}{mm}, portata \SI{20}{cm}];
	\item bilancia di precisione [portata: \SI{2000}{g}, incertezza: \SI{0.03}{g}];
\end{itemize}
\subsection{Campioni}
\begin{itemize}
	\item $10$ coppie di bulloni da inserire sul disco del Volano; 
	\item $2$ palette per aumentare l'attrito del mezzo.
\end{itemize}

%==================SEQUENZA OPERAZIONI SPERIMENTALI==========
\section{Sequenza Operazioni Sperimentali} 

\subsecion{Estrazione del momento della forza d'attrito e del momento d'inerzia}

Inserendo due coppie di bulloni per misura, senza compromettere la simmetria del sistema analizzato, studiamo il moto di discesa della massa $m$. Per ogni coppia di bulloni sono state acquisite $10$ misure di velocità in funzione del tempo. Le diverse stime dell'accelerazione sono ottenute dal coeficente angolare dei grafici di tali velocità. Ognuna delle misure di accelerazione così ottenute è affetta da un incertezza fornitaci dal programma \emph{Data Studio}. Poichè queste incertezze sono disomogenee, per tenerne conto realizziamo una media pesata per ogni raccolta. I dati sono riportati nella tabella (inserire riferimento) con le rispettive incertezze, ottenute sommando in quadratura l'incertezza da associare alla media pesata (per definizione $\frac{1}{\sqrt{\sum_{i}p_i}}$, con $p_i = \frac{1}{\sigma_{a_i}^2}$) e la deviazione standard della media ($\sigma/\sqrt{10}$).
Poichè l'inverso dell'accelerazione e il numero di bulloni inseriti sono legati da una relazione lineare è possibile ottenere una stima del momento d'inerzia del disco del volano, $I_0$, e del momento dell'attrito radente, $M_a$. Partendo dalle equazioni cardinali della dinamica otteniamo tale relazione per l'inverso dell'accelerazione: 
	
	\begin{equation}\label{eq:equazioneretta}
	\frac{1}{a}= A + B\cdot n
	\end{equation}
	
dove $n$ è il numero di coppie di bulloni, mente \emph{$A$} e \emph{$B$} sono rispettivamente :
	
	\begin{equation}\label{eq:intercettaecoefficente}
	A = \frac{mr^2+I_0}{mgr^2-M_a r}; \qquad B = \frac{2m_b R^2}{mgr^2-m_a r}.
	\end{equation}

Una volta sitmati, utilizzado il metodo dei minimi quadrati, intercetta ($A$) e coeficente angolare ($B$) della retta che meglio approssima i punti sperimentali, ricaviamo:

\begin{equation}\label{eq:momentoattrito}
M_a = mgr - \frac{2m_b R^2}{Br}
\end{equation}

\begin{equation}\label{eq:momentoinerzia}
I_0 = 2m_bR^2 \frac{A}{B} - mr^2
\end{equation}

Dove i parametri rappresentano rispettivamente:
\begin{itemize}
	\item $m$ :massa appesa al filo inestensibile del volano 
	\item $m_b$ :media delle masse dei $16$ bulloni 
	\item $r$ :raggio della puleggia 
	\item $R$ :distanza tra l'asse di rotazione del volano e i bulloni 
	\item $g$ :accelerazione di gravità 
\end{itemize}


Gli errori da associare alla stima di $M_a$ e $I_0$ sono dati dalla propagazione delle incertezze. Derivando parzialmente rispetto ai parametri delle quazioni \ref{eq:momentoattrito} e \ref{eq:momentoinerzia} otteniamo:

\begin{equation}\label{eq:incertezzamomentoattrito}
\sigma_{M_a}= M_a \cdot \sqrt{ \frac {\partial{M_a}}{ \partial m} ^2 \cdot \sigma_m^2 + \frac {\partial{M_a}}{ \partial r} ^2 \cdot \sigma_r^2 + \frac {\partial{M_a}}{ \partial R} ^2 \cdot \sigma_R^2 + \frac {\partial{M_a}}{ \partial m_b} ^2 \cdot \sigma_{m_b}^2 + \frac {\partial{M_a}}{ \partial B} ^2 \cdot \sigma_{B}^2 }
\end{equation}

Dove con $\sigma_m$,  $\sigma_R$, $\sigma_r$ e $\sigma_{m_b}$ intendiamo l'incertezza sulla massa, sulla distanza tra asse di rotazione e bulloni, sul raggio della puleggia e sulla massa dei bulloni date dalle rispettive incertezze strumentali. Con $\sigma_B$ indichiamo ivece l'errore sul coeficiente angolare stimato con il metodo dei mini quadrati. \newline
Derivando parzialmente si ottiene:

\begin{align*}\label{eq:parziali}
\frac{\partial{M_a}}{ \partial m} &=g\cdot r \\
\frac{\partial{M_a}}{ \partial r} &=mg + \frac{2m_b R^2}{Br^2}\\
\frac{\partial{M_a}}{ \partial R} &=-\frac{4m_bR}{Br} \\
\frac{\partial{M_a}}{ \partial B} &=\frac{2m_b R^2}{B^2 r}\\
\frac{\partial{M_a}}{ \partial m_b} &=-\frac{2R^2}{Br} \\
\end{align*}


L'errore sul momento d'inerzia del volano è ottenuto con lo stesso procedimento, per cui: 
\begin{equation}\label{eq:incertezzamomentoinerzia}
\sigma_{I_0} = I_0 \cdot \sqrt{ \frac {\partial{I_0}}{ \partial m} ^2 \cdot \sigma_m^2 + \frac {\partial{I_0}}{ \partial r} ^2 \cdot \sigma_r^2 + \frac {\partial{I_0}}{ \partial R} ^2 \cdot \sigma_R^2 + \frac {\partial{I_0}}{ \partial m_b} ^2 \cdot \sigma_{m_b}^2+ \frac {\partial{I_0}}{\partial B} ^2 \cdot \sigma_{B}^2 + \frac {\partial{I_0}}{\partial A} ^2 \cdot \sigma_{A}^2 + 2 \frac{\partial M_a}{\partial B} \frac{\partial M_a}{\partial A} cov(A, B)}
\end{equation} 

 Dove le incertezze sono le stesse citate precedentemente, eccetto per $\sigma_A$ che indica l'errore sull'intercetta anch'esso stimato con il metodo dei minimi quadrati. In questo caso è necessario anche tener conto della correlazione tra i parametri $A$ e $B$, perciò viene inserito anche il termine di covarianza. Derivando si ottiene: 
 \begin{align*}\label{eq:partialI}
 \frac{\partial{I_0}}{ \partial m} &=-r^2 \\
 \frac{\partial{I_0}}{ \partial r} &=-2mr\\
 \frac{\partial{I_0}}{ \partial R} &= 4Rm_b \frac{A}{B}\\
 \frac{\partial{I_0}}{ \partial m_b} &= 2R^2 \frac{A}{B}\\
 \frac{\partial{I_0}}{ \partial B} &= -\frac{2m_bR^2 A}{B^2}\\
 \frac{\partial{I_0}}{ \partial A} &=\frac{2m_bR^2}{B} \\
\end{align*}

I valori dei parametri $A$ e $B$ ottenuti sono: 

\begin{equation}
\boxed{\bf{A = \pm }}
\end{equation}
\begin{equation}
\boxed{\bf{B = \pm }}
\end{equation}

Nella figura (inserire riferimento) è riportato il grafico di $\frac{1}{a}$ in funzione del numero di coppie di bulloni. I valori di intercetta e coeficiente angolare della retta forniti dal softwer di analisi dati \emph{R} sono: 
\begin{equation}
B = 5.511 \qquad A = 19.750
\end{equation}
che risultano compatibili con le stime realtizzate manualmente di tali parametri. 

La migliore stima di $M_a$ sarà quindi: 
\begin{equation}
\boxed{\bf{M_a = \pm }}
\end{equation}

Mentre la migliore stima di $I_0$ è: 
\begin{equation}
\boxed{\bf{I_0 = \pm }}
\end{equation}



\subsecion{Studio della perdita di energia potenziale gravitazionale a causa dell'attrito}

\subsection{studio del moto del volano in presenza di attrito del mezzo}

%========================= CONCLUSIONI =============
\section{considerazioni Finali}

\pagebreak
%=================APPENDICE==================================

\section{Appendice: tabelle e grafici}


\end{document}
